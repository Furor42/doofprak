% Der Befehl \newcommand kann auch benutzt werden um Variablen zu definieren:

% Nummer laut Praktikumsheft:
    \newcommand{\varNum}{M1}
% Name laut Praktikumsheft:
    \newcommand{\varName}{Pendel}
% Datum der Durchführung:
    \newcommand{\varDate}{03.11.2014}
% Autoren des Protokolls:
    \newcommand{\varAutor}{Christian Mannweiler, Robin Balske}
% Nummer der eigenen Gruppe (z.B. "1mo"):
    \newcommand{\varGruppe}{Gruppe 5}
% E-Mail-Adressen der Autoren:
    \newcommand{\varEmail}{christian.mannweiler@uni-muenster.de \\ r\_bals02@wwu.de}
% E-Mail-Adresse anzeigen (true/false):
    \newcommand{\varZeigeEmail}{true}
% Literaturverzeichnis anzeigen (true/false):
    \newcommand{\varZeigeLiteraturverzeichnis}{true}
% Stil der Einträge im Literaturverzeichnis
    \newcommand{\varLiteraturLayout}{unsrtdin}
